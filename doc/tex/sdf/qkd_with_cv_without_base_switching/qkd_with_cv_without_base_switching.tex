\clearpage
\section{Quantum Key Distribution Without Basis Switching}

\begin{refsection}

\begin{tcolorbox}	
\begin{tabular}{p{2.75cm} p{0.2cm} p{10.5cm}} 	
\textbf{Student Name}  &:&  Goncalo Nunes (2018/05/20- )\\
\textbf{Goal}          &:& Provide practical evidence of the employability of a coherent state quantum key distribution protocol based on simultaneous quadrature measurements.\\
\textbf{Directory}              &:& sdf/qkd\_with\_cv\_without\_base\_switching
\end{tabular}
\end{tcolorbox}Quantum key distribution (QKD) is a method to generate a cryptographic key between two distant parties, Alice and Bob, based on transmission of quantum states. After said transmission and respective measurement, Alice and Bob can then exchange classical messages through an insecure channel and, using the keys generated, perform post-processing to recover the messages.\\ 
In the first quantum key distribution schemes, single photons acted as information carriers (discrete variable regime). The practical implementation of said schemes was limited by the single photon generation and detection techniques. Hence the current interest in continuous variable (CV) quantum cryptography, which allows for higher key rates. The security of a CV-QKD relies on randomly switching the measurement basis (quadratures). In practice this implies a change of phase of a local oscillator beam which is difficult to achieve and constitutes a serious technical difficulty for the implementation of this type of cryptosystem, compromising its bandwidth.\\
Hence the pertinence of a QKD scheme that doesn't require for the change in measurement basis. Such protocol was analysed in \cite{Weedbrook2004} and is called the simultaneous quadrature measurement or SQM Protocol, where both bases are measured simultaneously through double homodyne detection. It utilizes the quantum channel more effectively and achieves both higher secret key rates and bandwidths compared to orthodox CV-QKD protocols.
\subsection{Theoretical Analysis}
The scheme is similar to an ordinary continuous variable coherent state quantum cryptography protocol. Alice prepares a state by displacing the amplitude and phase quadratures of a vacuum state according to a random Gaussian variable. The quadrature
operators of Alice’s state are given by:\\
\begin{equation}
\hat{X}^\pm_A=S^\pm+\hat{X}^\pm
\end{equation} 
Where $S^\pm$ are the quadrature operators of the initial vacuum state \cite{Weedbrook2004}. In practice this is done through the modulation of previously prepared squared laser pulse using phase and amplitude modulation, changing the quadratures $\hat{X}^.$ and $\hat{X}^+$ respectively. Afterwards Alice sends a multiplexed signal: one polarization containing the modulated signal and the orthogonal one with a reference, commonly named local oscillator. This multiplexed signal is mixed using a polarization beam splitter, PBS. Unwanted feedback in the circuit can be prevented by implementing optical isolators, as displayed in the scheme.\\
The multiplexed signal is sent through a quantum channel, an optical fiber, to Bob with efficiency $\eta$ and couples in channel noise $N^\pm$. The signal is demultiplex using again a PBS, splitting the modulated state and the reference. Bob simultaneously measures the amplitude and phase quadratures of the state using a 50/50 beam splitter. Part of the split local oscillator's phase is shifted $\pi/2$ for the measurement of $\hat{X}^+$ and is left unchanged for $\hat{X}^-$. The quadratures are measured through the comparison of the modulated signal with the local oscillator, this is homodyne detection.\\
The state Bob received is given by \cite{Weedbrook2004},
\begin{equation}
\hat{X}_B^\pm=\frac{1}{\sqrt{2}}\left(\sqrt{\eta}\hat{X}^\pm_A+\sqrt{1-\eta}\hat{X}^\pm_N+\hat{N}_B^\pm\right)
\end{equation}
Alice and Bob then test the channel transmission and bit error rate of their key by using a randomly chosen subset of their key to check for errors. If the errors are within some tolerated limit, they then use secret key distillation with the objective to produce a key between Alice and Bob which has negligible errors. The rate at which the key is generated has a lower bound given by \cite{Weedbrook2004}:\\
\begin{equation}
\text{Lower Bound}=\log_2\left(\frac{\left(\frac{\eta}{V_A}+\left(1-\eta\right)V_N\right)^{-1}+1}{\eta+\left(1-\eta\right)V_N+1}\right)
\end{equation}
where $V_A$ is the variance of Alice's state and symmetry between the amplitude and phase quadratures is assumed. We see that, so long as the channel noise $V_N$ is not excessive, a secret key can be successfully generated between Alice and Bob, even in the limit of very small channel efficiency $\eta$.  
%\begin{figure}[h]
%\centering
%\includegraphics[width=.4\linewidth]{./sdf/bpsk_system/figures/bpskconstellation.jpg}
%\caption{BPSK symbol constellation.}
%\label{fig:BPSKConst}
%\end{figure}

% bibliographic references for the section ----------------------------
\clearpage
\printbibliography[heading=subbibliography]
\end{refsection}
\addcontentsline{toc}{subsection}{Bibliography}
\cleardoublepage
% --------------------------------------------------------------------- 