\clearpage
\section{Quantum Random Number Generator}

\begin{tcolorbox}	
\begin{tabular}{p{2.75cm} p{0.2cm} p{10.5cm}} 	
\textbf{Students Name}  &:& Mariana Ramos\\
\textbf{Starting Date} &:& January 12, 2018\\
\textbf{Goal}          &:& sdf/quantum\_random\_number\_generator.
\end{tabular}
\end{tcolorbox}

True random numbers are indispensable in the field of cryptography to guarantee the security of the communication protocols. There are two approaches for random number generation, which in some applications must be unpredictable: the pseudorandom generation which are based on some kind of classical algorithms implemented on a computing device, or the physical random generators which consist in measuring some physical observable with random behaviour. 

In this chapter, it is presented the theoretical analysis and simulation of a quantum random generator by measuring the polarization of single photons with a polarizing beam splitter.

\subsection{Theoretical Analysis}

Nowadays, the only known way to generate truly random numbers is by building a physical source by using quantum mechanical decisions, since the occurrence of each individual result of such a quantum mechanical decision is truly random, ie it is inestimable or unknowable. One of the optical processes available as a source of randomness is the splitting of a polarized single photon beam.



\subsection{Simulation Analysis}



\subsection{Open Issues}



\cleardoublepage
