\clearpage
\section{BPSK Transmission System}

\begin{refsection}

\begin{tcolorbox}	
\begin{tabular}{p{2.75cm} p{0.2cm} p{10.5cm}} 	
\textbf{Student Name}  &:&  Andr\'e Mourato (2018/01/28 - 2018/02/27)\\
\textbf{}  & &  Daniel Pereira (2017/09/01 - 2017/11/16)\\
\textbf{Goal}          &:& Estimate the BER in a Binary Phase Shift Keying optical transmission system with additive white Gaussian noise. Comparison with theoretical results.\\
\textbf{Directory}              &:& sdf/bpsk\_system
\end{tabular}
\end{tcolorbox}

Binary Phase Shift Keying (BPSK) is the simplest form of Phase Shift Keying (PSK), in which binary information is encoded into a two state constellation with the states being separated by a phase shift of $\pi$ (see Figure~\ref{fig:BPSKConst}).

\begin{figure}[h]
\centering
\includegraphics[width=.4\linewidth]{./sdf/bpsk_system/figures/bpskconstellation.jpg}
\caption{BPSK symbol constellation.}
\label{fig:BPSKConst}
\end{figure}

\par
White noise is a random signal with equal intensity at all frequencies, having a constant power spectral density. White noise is said to be Gaussian (WGN) if its samples follow a normal distribution with zero mean and a certain variance $\sigma^2$. For WGN its spectral density equals its variance. For the purpose of this work, additive WGN is used to model thermal noise at the receivers.
\par
The purpose of this system is to simulate BPSK transmission in back-to-back configuration with additive WGN at the receiver and to perform an accurate estimation of the BER and validate the estimation using theoretical values.

\subsection{Theoretical Analysis}

The output of the system with added gaussian noise follows a normal distribution, whose first probabilistic moment can be readily obtained by knowledge of the optical power of the received signal and local oscillator,
\begin{equation}\label{eq:mu1}
m_i=2\sqrt{P_LP_S}G_{ele}\cos(\Delta\theta_i),
\end{equation}
where $P_L$ and $P_S$ are the optical powers, in watts, of the local oscillator and signal, respectively, $G_{ele}$ is the gain of the trans-impedance amplifier in the coherent receiver and $\Delta\theta_i$ is the phase difference between the local oscillator and the signal, for BPSK this takes the values $\pi$ and 0, in which case~\eqref{eq:mu1} can be reduced to,
\begin{equation}
m_i=(-1)^{i+1}2\sqrt{P_LP_S}G_{ele},~i=0,~1.
\end{equation}
The second moment is directly chosen by inputting the spectral density of the noise $\sigma^2$, and thus is known \textit{a priori}.
\par
Both probabilist moments being known, the probability distribution of measurement results is given by a simple normal distribution,
\begin{equation}
f(x)=\frac{1}{\sqrt{2\pi}\sigma}e^{-\frac{(x-m_i)^2}{2\sigma^2}}.
\end{equation}
The BER is calculated in the following manner,
\begin{equation}
BER=\frac{1}{2}\int_0^{+\infty}f(x|\Delta\theta=\pi)\text{d}x+\frac{1}{2}\int^0_{-\infty}f(x|\Delta\theta=0)\text{d}x,
\end{equation}
given the symmetry of the system, this can be simplified to,
\begin{equation}\label{eq:BERtheoretical}
BER=\int_0^{+\infty}f(x|\Delta\theta=\pi)\text{d}x=\frac{1}{2}\text{erfc}\left(\frac{-m_0}{\sqrt{2}\sigma}\right)
\end{equation}

\begin{figure}[H]
\centering
\includegraphics[width=.7\linewidth]{./sdf/bpsk_system/figures/BER_Evolution_theo.pdf}
\caption{Bit Error Rate in function of the signal power in dBm at a constant local oscillator power level of 0~dBm.}
\label{fig:berevolution}
\end{figure}

\subsection{Simulation Analysis}

A diagram of the system being simulated is presented in the Figure~\ref{fig:homodynesystem}. A random binary sequence is generated and encoded in an optical signal using BPSK modulation. The decoding of the optical signal is accomplished by an homodyne receiver, which combines the signal with a local oscillator. The received binary signal is compared with the transmitted binary signal in order to estimate the Bit Error Rate (BER). The simulation is repeated for multiple signal power levels. Each corresponding BER is recorded and plotted against the expectation value.

\begin{figure}[h]
\centering
%left bottom right top
\includegraphics[width=\linewidth, trim={0.4cm 3.8cm 0.4cm 2.5cm}, clip=true]{BPSK_BW.pdf}
\caption{Overview of the BPSK system being simulated.}
\label{fig:homodynesystem}
\end{figure}

\subsection*{Required files}\label{Required files}

\begin{table}[H]
\centering
\begin{tabulary}{1.0\textwidth}{|p{7.5cm}|p{6.5cm}|p{1cm}|}
\hline
\multicolumn{3}{|c|}{ \textbf{Header Files} } \\
\hline
\textbf{File}                    & \textbf{Comments} & \textbf{Status} \\ \hline
add\_20171116.h                            &                   & \checkmark \\ \hline
balanced\_beam\_splitter\_20180124.h       &                   & \checkmark \\ \hline
binary\_source\_20180118.h                 &                   & \checkmark \\ \hline
bit\_decider\_20170818.h                   &                   & \checkmark \\ \hline
bit\_error\_rate\_20171810.h               &                   & \checkmark \\ \hline
discrete\_to\_continuous\_time\_20180118.h &                   & \checkmark \\ \hline
i\_homodyne\_receiver\_20180124.h          &                   & \checkmark \\ \hline
ideal\_amplifier\_20180118.h               &                   & \checkmark \\ \hline
iq\_modulator\_20180130.h                  &                   & \checkmark \\ \hline
local\_oscillator\_20180130.h              &                   & \checkmark \\ \hline
m\_qam\_mapper\_20180118.h                 &                   & \checkmark \\ \hline
m\_qam\_transmitter\_20180118.h            &                   & \checkmark \\ \hline
netxpto\_20180118.h                        &                   & \checkmark \\ \hline
photodiode\_old\_20180118.h                &                   & \checkmark \\ \hline
pulse\_shaper\_20180118.h                  &                   & \checkmark \\ \hline
sampler\_20171116.h                        &                   & \checkmark \\ \hline
sink\_20180118.h                           &                   & \checkmark \\ \hline
super\_block\_interface\_20180118.h        &                   & \checkmark \\ \hline
ti\_amplifier\_20180102.h                  &                   & \checkmark \\ \hline
white\_noise\_20180118.h                   &                   & \checkmark \\ \hline
\end{tabulary}
\end{table}		
%
\begin{table}[H]
\centering
\begin{tabulary}{1.0\textwidth}{|p{7.5cm}|p{6.5cm}|p{1cm}|}
\hline
\multicolumn{3}{|c|}{ \textbf{Source Files} } \\
\hline
\textbf{File}                      & \textbf{Comments} & \textbf{Status} \\ \hline
add\_20171116.cpp                            &                   & \checkmark \\ \hline
balanced\_beam\_splitter\_20180124.cpp       &                   & \checkmark \\ \hline
binary\_source\_20180118.cpp                 &                   & \checkmark \\ \hline
bit\_decider\_20170818.cpp                   &                   & \checkmark \\ \hline
bit\_error\_rate\_20171810.cpp               &                   & \checkmark \\ \hline
discrete\_to\_continuous\_time\_20180118.cpp &                   & \checkmark \\ \hline
i\_homodyne\_receiver\_20180124.cpp          &                   & \checkmark \\ \hline
ideal\_amplifier\_20180118.cpp               &                   & \checkmark \\ \hline
iq\_modulator\_20180130.cpp                  &                   & \checkmark \\ \hline
local\_oscillator\_20180130.cpp              &                   & \checkmark \\ \hline
m\_qam\_mapper\_20180118.cpp                 &                   & \checkmark \\ \hline
m\_qam\_transmitter\_20180118.cpp            &                   & \checkmark \\ \hline
netxpto\_20180118.cpp                        &                   & \checkmark \\ \hline
photodiode\_old\_20180118.cpp                &                   & \checkmark \\ \hline
pulse\_shaper\_20180118.cpp                  &                   & \checkmark \\ \hline
sampler\_20171116.cpp                        &                   & \checkmark \\ \hline
sink\_20180118.cpp                           &                   & \checkmark \\ \hline
super\_block\_interface\_20180118.cpp        &                   & \checkmark \\ \hline
ti\_amplifier\_20180102.cpp                  &                   & \checkmark \\ \hline
white\_noise\_20180118.cpp                   &                   & \checkmark \\ \hline
\end{tabulary}
\end{table}		

\subsection*{System Input Parameters}

This system takes into account the following input parameters:

\begin{table}[H]
\centering
\begin{tabulary}{1.0\textwidth}{|p{6cm}|p{4cm}|p{5cm}|}
\hline
\multicolumn{3}{|c|}{ \textbf{System Input Parameters} } \\
\hline
\textbf{Parameter}     & \textbf{Default Value}                                     & \textbf{Comments} \\ \hline
numberOfBitsReceived   & $-1$	                                                    & \\ \hline
numberOfBitsGenerated  & $1000$	                                                    & \\ \hline
samplesPerSymbol       & $16$                                                       & \\ \hline
pLength                & $5$                                                        & \\ \hline
bitPeriod              & $20\times10^{-12}$                                         & \\ \hline
rollOffFactor          & $0.3$                                                      & \\ \hline
signalOutputPower\_dBm & $-20$	                                                    & \\ \hline
localOscillatorPower\_dBm & $0$                                                     & \\ \hline
localOscillatorPhase   & $0$                                                        & \\ \hline
iqAmplitudesValues     & $\lbrace~\lbrace-1,~0\rbrace~,~\lbrace1,~0\rbrace~\rbrace$ & \\ \hline
transferMatrix         & $\lbrace~\lbrace \frac{1}{\sqrt{2}},~\frac{1}{\sqrt{2}},~\frac{1}{\sqrt{2}},~\frac{-1}{\sqrt{2}} \rbrace~\rbrace$ & \\ \hline
responsivity           & $1$                                                        & \\ \hline
amplification          & $10^6$                                                     & \\ \hline
electricalNoiseAmplitude  & $5\times10^{-4}\sqrt{2}$                                & \\ \hline
samplesToSkip          & $8\times samplesPerSymbol$                                 & \\ \hline
bufferLength           & $20$                                                       & \\ \hline
shotNoise              & false                                                      & \\ \hline
\end{tabulary}
\end{table}		

\subsection*{Inputs}

This system takes no inputs.

\subsection*{Outputs}

This system outputs the following objects:
\begin{table}[H]
\centering
\begin{tabulary}{1.0\textwidth}{|p{8cm}|p{7cm}|}
\hline
\multicolumn{2}{|c|}{ \textbf{System Output Signals} } \\
\hline
\textbf{Signal}     & \textbf{Associated File} \\ \hline
Initial Binary String (S$_0$) & S0.sgn \\ \hline
Optical Signal with coded Binary String (S$_{1}$) & S1.sgn \\ \hline
Local Oscillator Optical Signal (S$_{2}$) & S2.sgn \\ \hline
Beam Splitter Outputs (S$_{3}$, S$_{4}$) & S3.sgn \& S4.sgn \\ \hline
Homodyne Receiver Electrical Output (S$_{5}$) & S5.sgn \\ \hline
Sampler Output (S$_{6}$) & S6.sgn \\ \hline
Decoded Binary String (S$_{7}$) & S7.sgn \\ \hline
BER Result String (S$_{8}$) & S8.sgn \\ \hline
\textbf{Report}     & \textbf{Associated File} \\ \hline
Bit Error Rate Report & BER.txt \\ \hline
\end{tabulary}
\end{table}	
\pagebreak
\subsection*{Bit Error Rate - Simulation Results}
\begin{figure}[H]
\centering
\includegraphics[width=.7\linewidth]{./sdf/bpsk_system/figures/BER_Evolution_simu.pdf}
\caption{Bit Error Rate in function of the signal power in dBm at a constant local oscillator power level of 0~dBm.}
\label{fig:berevolution}
\end{figure}

\subsection{Comparative Analysis}

The following results show the dependence of the error rate with the signal power assuming a constant Local Oscillator power of $0~dBm$, the signal power was evaluated at levels between -70~and~-25~dBm, in steps of 5~dBm between each. The simulation results are presented in orange with the computed lower and upper bounds, while the expected value, obtained from~\eqref{eq:BERtheoretical}, is presented as a full blue line. A close agreement is observed between the simulation results and the expected value. The noise spectral density was set at $5\times10^{-4}\sqrt{2}$~V$^2$~\cite{thorlabs}.
\begin{figure}[H]
\centering
\includegraphics[width=.7\linewidth]{./sdf/bpsk_system/figures/BER_Evolution.pdf}
\caption{Bit Error Rate in function of the signal power in dBm at a constant local oscillator power level of 0~dBm. Theoretical values are presented as a full blue line while the simulated results are presented as a errorbar plot in orange, with the upper and lower bound computed in accordance with the method described in~\ref{bercalc}}
\label{fig:berevolution}
\end{figure}


% bibliographic references for the section ----------------------------
\clearpage
\printbibliography[heading=subbibliography]
\end{refsection}
\addcontentsline{toc}{subsection}{Bibliography}
\cleardoublepage
% --------------------------------------------------------------------- 