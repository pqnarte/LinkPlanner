%\documentclass[a4paper]{article}
%\usepackage[top=1in, bottom=1.25in, left=1.25in, right=1.25in]{geometry}
%\usepackage{amsmath}
%\usepackage{multicol}
%\usepackage{graphicx}
%\usepackage{subfig}
%\usepackage{amssymb}
%%\RequirePackage{ltxcmds}[2010/12/07]
%%opening
%\title{Linear Filtering in Frequency-Domain}
%\author{ }
%\date{ }
%\begin{document}
%
%\maketitle
%Python code highlighting
\definecolor{mygreen}{rgb}{0,0.6,0}
\definecolor{mygray}{rgb}{0.5,0.5,0.5}
\definecolor{mymauve}{rgb}{0.58,0,0.82}

\lstset{ %
	backgroundcolor=\color{white},   % choose the background color; you must add \usepackage{color} or \usepackage{xcolor}; should come as last argument
	basicstyle=\footnotesize,        % the size of the fonts that are used for the code
	breakatwhitespace=false,         % sets if automatic breaks should only happen at whitespace
	breaklines=true,                 % sets automatic line breaking
	captionpos=b,                    % sets the caption-position to bottom
	commentstyle=\color{mygreen},    % comment style
	deletekeywords={...},            % if you want to delete keywords from the given language
	escapeinside={\%*}{*)},          % if you want to add LaTeX within your code
	extendedchars=true,              % lets you use non-ASCII characters; for 8-bits encodings only, does not work with UTF-8
	frame=single,	                   % adds a frame around the code
	keepspaces=true,                 % keeps spaces in text, useful for keeping indentation of code (possibly needs columns=flexible)
	keywordstyle=\color{blue},       % keyword style
	language=Octave,                 % the language of the code
	morekeywords={*,...},            % if you want to add more keywords to the set
	numbers=left,                    % where to put the line-numbers; possible values are (none, left, right)
	numbersep=5pt,                   % how far the line-numbers are from the code
	numberstyle=\tiny\color{mygray}, % the style that is used for the line-numbers
	rulecolor=\color{black},         % if not set, the frame-color may be changed on line-breaks within not-black text (e.g. comments (green here))
	showspaces=false,                % show spaces everywhere adding particular underscores; it overrides 'showstringspaces'
	showstringspaces=false,          % underline spaces within strings only
	showtabs=false,                  % show tabs within strings adding particular underscores
	stepnumber=2,                    % the step between two line-numbers. If it's 1, each line will be numbered
	stringstyle=\color{mymauve},     % string literal style
	tabsize=2,	                   % sets default tabsize to 2 spaces
	title=\lstname                   % show the filename of files included with \lstinputlisting; also try caption instead of title
}
\clearpage
\section{Hilbert Transform}
\begin{tcolorbox}	
	\begin{tabular}{p{2.75cm} p{0.2cm} p{10.5cm}} 	
		\textbf{Header File}   &:& hilbert\_filter\_*.h \\
		\textbf{Source File}   &:& hilbert\_filter\_*.cpp \\
		\textbf{Version}       &:& 20180306 (Romil Patel)
	\end{tabular}
\end{tcolorbox}
\subsection*{What is the purpose of Hilbert transform?}
The Hilbert transform facilitates the formation of analytical signal. An analytic signal is a complex-valued signal that has no negative frequency components, and its real and imaginary parts are related to each other by the Hilbert transform.
\begin{equation}
s_a(t)=s(t)+i\hat{s}(t)
\label{Analytical signal}
\end{equation}
where, $s_a(t)$ is an analytical signal and $\hat{s}(t)$ is the Hilbert transform of the signal ${s}(t)$. Such analytical signal can be used to generate Single Sideband Signal (SSB) signal.
\subsection*{Transfer function for the discrete Hilbert transform}
There are two approached to generate the analytical signal using Hilbert transformation method. First method generates the analytical signal $S_a(t)$ directly, on the other hand, second method will generate the $\hat{s}(t)$ signal which is multiplied with $i$ and added to the ${s}(t)$ to generate the analytical signal $S_a(t)$.\\
\textbf{Method 1 :}\\
The discrete time analytical signal $S_a(t)$ corresponding to  ${s}(t)$ is defined in the frequency domain as
\begin{equation}
S_a(f) = \begin{cases}
2S(f) &\text{for $f>0$}\\
S(f) &\text{for $f=0$}\\
0 &\text{for $f<0$}
\end{cases}
\end{equation}
which is inverse transformed to obtain an analytical signal $S_a(t)$.\\
\textbf{Method 2 :}\\
The discrete time Hilbert transformed signal  $\hat{s}(t)$ corresponding to  ${s}(t)$ is defined in the frequency domain as
\begin{equation}
 \hat{S}(f)= \begin{cases}
-i S(f) &\text{for $f>0$}\\
0 &\text{for $f=0$}\\
i S(f) &\text{for $f<0$}
\end{cases}
\end{equation}
which is inverse transformed to obtain a Hilbert transformed signal $\hat{S}(t)$. To generate an analytical signal,  $\hat{S}(t)$ is added to the $S(t)$ to get the equation \ref{Analytical signal}.\\
\subsection*{Real-time Hilbert transform : Proposed logical flow}
To understand the new proposed method, consider that the signal consists of 2048 samples and the \textbf{bufferLength} is 512. Therefore, by considering the \textbf{bufferLength}, we will process the whole signal in four consecutive blocks namely $A$, $B$, $C$ and $D$; each with the length of 512 samples as shown in Figure \ref{data}.
\begin{figure}[h]
	\centering
	\includegraphics[width=14cm]{./algorithms/hilbert/figures/data.pdf}
	\caption{Logical flow}
	\label{data}
\end{figure}
The filtering process will start only after acquiring first two blocks $A$ and $B$ (see iteration 1 in Figure \ref{Logical flow}), which introduces delay in the system. In the iteration 1, $x(n)$ consists of 512 front Zeros, block $A$ and block $B$ which makes the total length of the $x(n)$ is $512$ x $3 = 1536$ symbols. After applying filter to the $x(n)$, we will capture the data which corresponds to the block $A$ only and discard the remaining data from each side of the filtered output.\\
\begin{figure}[h]
	\centering
	\includegraphics[width=14cm]{./algorithms/hilbert/figures/proposedLogic.pdf}
	\caption{Logical flow of real-time Hilbert transform}
	\label{Logical flow}
\end{figure}
In the next iteration 2, we'll use \textbf{previousCopy} $A$ and $B$ along with the \textbf{currentCopy} "C" and process the signal same as we did in iteration  and we will continue the procedure until the end of the sequence.
\subsection*{Real-time Hilbert transform : Test setup}
\begin{figure}[h]
	\centering
	\includegraphics[width=14cm]{./algorithms/hilbert/figures/test_setup.pdf}
	\caption{Test setup for the real time Hilbert transform }\label{test_setup}
\end{figure}
\subsection*{Real-time Hilbert transform : Results}
\begin{figure}[h]
	\centering
	\includegraphics[width=13cm]{./algorithms/hilbert/figures/S5.eps}
	\caption{Complex analytical signal $S5$ with its real part $S4$ and imaginary part $\hat{S}4$}\label{S5}
\end{figure}
\textbf{Remark :} Here, we have used method 1 to generate analytical signal using Hilbert transform.


%\subsection*{Remarks}
%Processing the data with the proposed overlap save method introduce delay of 512 samples (more precisely, it is equal to the bufferLength) and hence we cannot capture the last 512 samples of the input signal. \\ \textbf{Solution :} We need one flag which indicates the end of the input signal so that we can and acquire the full signal (see iteration 3 in the Figure \ref{Logical flow} how the last block has been recovered).
%%%%%%%%%%%%%%%%%%%%%%%%%%%%%%%%%%%%%%%%%%%%%%%%%%%%%%%%%%%%%%%%%%%%%%%%%%%%%%%
%%%%%%%%%%%%%%%%%%%%%%%%%%%%%%%%%%%%%%%%%%%%%%%%%%%%%%%%%%%%%%%%%%%%%%%%%%%%%%%%
%%%%%%%%%%%%%%%%%%%%%%%%%%%%%%%%%%%%%%%%%%%%%%%%%%%%%%%%%%%%%%%%%%%%%%%%%%%%%%%
%%%%%%%%%%%%%%%%%%%%%%%%%%%%%%%%%%%%%%%%%%%%%%%%%%%%%%%%%%%%%%%%%%%%%%%%%%%%%%%









%\renewcommand{\bibname}{References}
%
%\bibliographystyle{myIEEEtran}
%\bibliography{./algorithms/fft/fft}
%
%

\cleardoublepage


%%%%%%%%%%%%%%%%%%%%%%%%%%%%%%%%%%%%%%%%%%%%%%%%%%%%%
%%%%%%%%%%% Example flowchart START %%%%%%%%%%%%%%%%%
%\tikzstyle{decision} = [diamond, draw, fill=blue!20,
%text width=4.5em, text badly centered, node distance=3cm, inner sep=0pt]
%\tikzstyle{block} = [rectangle, draw, fill=blue!20,
%text width=5em, text centered, rounded corners, minimum height=4em]
%\tikzstyle{line} = [draw, -latex']
%\tikzstyle{cloud} = [draw, ellipse,fill=red!20, node distance=3cm,
%minimum height=2em]
%\begin{tikzpicture}[node distance = 2cm, auto]
%% Place nodes
%\node [block] (init) {initialize model};
%\node [cloud, left of=init] (expert) {expert};
%\node [cloud, right of=init] (system) {system};
%\node [block, below of=init] (identify) {identify candidate models};
%\node [block, below of=identify] (evaluate) {evaluate candidate models};
%\node [block, left of=evaluate, node distance=3cm] (update) {update model};
%\node [decision, below of=evaluate] (decide) {is best candidate better?};
%\node [block, below of=decide, node distance=3cm] (stop) {stop};
%% Draw edges
%%\path [line] (init) -- (identify);
%\path [line] (identify) -- (evaluate);
%\path [line] (evaluate) -- (decide);
%\path [line] (decide) -| node [near start] {yes} (update);
%\path [line] (update) |- (identify);
%\path [line] (decide) -- node {no}(stop);
%\path [line,dashed] (expert) -- (init);
%\path [line,dashed] (system) -- (init);
%\path [line,dashed] (system) |- (evaluate);
%\end{tikzpicture}
%%%%%%%%%%% Example flowchart END %%%%%%%%%%%%%%%%%
%%%%%%%%%%%%%%%%%%%%%%%%%%%%%%%%%%%%%%%%%%%%%%%%%%%


%\end{document}

