\clearpage
\section{BB84 with Discrete Variables}

\begin{tcolorbox}	
\begin{tabular}{p{2.75cm} p{0.2cm} p{10.5cm}} 	
\textbf{Students Name}  &:& Mariana Ramos and Kevin Filipe\\
\textbf{Starting Date} &:& November 7, 2017\\
\textbf{Goal}          &:& BB84 implementation with discrete variables.
\end{tabular}
\end{tcolorbox}

BB84 is a key distribution protocol.

\begin{enumerate}
	\item{The protocol must be... .}
	\item{The protocol ....}
\end {enumerate}

\subsection{BB84 Protocol Detailed}

\begin{table}[H]
\centering
\begin{tabular}{c|c}
\textbf{\textit{Basis}}         &  \\ \hline
 0 & $+$ \\
 1 & $\times$ \\
\end{tabular}
\end{table}


\begin{table}[H]
\centering
\begin{tabular}{c|c}
            & Basis "+" \\ \hline
 0 & $\to (0^{\circ})$ \\
 1 & $\uparrow (90^{\circ})$ \\
\end{tabular}
\end{table}


\begin{table}[H]
\centering
\begin{tabular}{c|c}
      & Basis "$\times$" \\ \hline
 0 & $\searrow (-45^{\circ})$ \\
 1 & $\nearrow (45^{\circ})$ \\
\end{tabular}
\end{table}


\begin{enumerate}
  \item Alice randomly generate a bit sequence with length $ks$ being, in this case, $k=2$ and $s=4$ as it was defined at the beginning.
      Therefore, she must define two sets randomly: $S_{A1}$ which contains the basis values; and $S_{A2}$, which contains the key values.

      In that case, lets assume she gets the following sets $S_{A1}$ and $S_{A2}$:
      $$S_{A1} = \{0,1,1,0,0,1,0,1 \},$$
      $$S_{A2} = \{1,1,0,0,0,1,0,0 \}.$$

  \item Next, Alice sends to Bob throughout a quantum channel $ks$ photons encrypted using the basis defined in $S_{A1}$ and according to the keys defined in $S_{A2}$.

      In the current example, Alice sends the photons, throughout a quantum channel, according to the following,

      $$S_{AB} = \{\uparrow, \nearrow, \searrow, \to, \to, \nearrow, \to, \searrow \}.$$
       $$S_{AB} = \{90^{\circ}, 45^{\circ}, -45^{\circ}, 0^{\circ}, 0^{\circ}, 45^{\circ}, 0^{\circ}, -45^{\circ} \}.$$

  \item Bob also randomly generates $ks$ bits, which are going to define his measurement basis, $S_{B1}$. He will measure the photons sent by Alice. Lets assume:

  $$S_{B1} = \{0,1,0,1,0,1,1,1 \}.$$

    When Bob receives photons from Alice, he measures them using the basis defined in $S_{B1}$.
  In the current example, $S_{B1}$ corresponds to the following set:
  $$\{ +,\times,+,\times,+,\times, \times, \times \}.$$
  Bob will get $ks$ results:
  $$S_{B1'} = \{1,1,0,1,0,1,1,0 \}.$$

  \item Bob will send a \textit{Hash Function} result HASH1 to Alice. This value will do Bob's commitment with the measurements done. In this case, this \textit{Hash Function} is calculated from \textit{SHA-256} algorithm for each pair (Basis from $S_{B1}$ and measured value from $S_{B1'}$), i.e Bob sends to Alice $sk$ pairs as his commitment. In this case, Bob sends eight pairs encoded using a \textit{Hash Function} which is also send to Alice. From that moment on Bob cannot change his commitment neither the basis which he uses to measure the photons sent by Alice.

  \item Once Alice has received the confirmation of measurement from Bob, she sends throughout a classical channel the basis which she has used to codify the photons, which in this case we assumed $S_{A1} = \{0,1,1,0,0,1,0,1\}$.

  \item In order to know which photons were measured correctly, Bob does the operation $S_{B2}=S_{B1} \oplus S_{A1}$.
      In the current example the operation will be:

  \begin{table}[H]
    \centering
    \begin{tabular}{c|c c c c c c c c}
     $S_{B1}$ & 0 & 1 & 0 & 1 & 0 & 1 & 1 & 1 \\
     $S_{A1}$ & 0 & 1 & 1 & 0 & 0 & 1 & 0 & 1 \\ \hline
     $\oplus$ & 1 & 1 & 0 & 0 & 1 & 1 & 0 & 1
    \end{tabular}
    \end{table}

      In this way, Bob gets $$S_{B2} = \{1,1,0,0,1,1,0,1 \}.$$ When Bob uses the right basis he gets the values correctly, when he uses the wrong basis he just guess the value. The values ``$1$'' correspond to the values he measured correctly and ``$0$'' to the values he just guessed.

       Next, Bob sends to Alice, through a classical channel, information about the minimum number between ``ones'' and ``zeros'', i.e $$n=min(\#0,\#1)=3,$$ where $\#0$ represents the number of zeros in $S_{B2}$ and $\#1$ the number of ones in $S_{B2}$. At this time, Alice must be able to know if Bob is being honest or not. Therefore, she will open Bob's commitment from \textit{step 4} and she verify if the number n sent by Bob is according with the commitment values sent by him. In other words, she opens a number of pairs committed by Bob which is known from the beginning.

  \item If $n<s$, being \textit{s} the message's size, Alice and Bob will repeat the steps from $1$ to $7$. In this case, $n=3$ which is smaller than $s=4$. Therefore, Alice and Bob repeat the steps from 1 to 7 in order to enlarge Bob's sets $S_{B1}$ and $S_{B2}$ as well as Alice's sets $S_{A1}$ and $S_{A2}$.

  \item Lets assume :

   $$S_{B1}= \{1,1,0,0,0,1,0,0,1,0,0,0,0,0,1,1 \}.$$

    At Alice's side the new sets $S_{A1}$, which contains the basis values, and $S_{A2}$, which contains the key values, will be the following:

    $$S_{A1}=\{0,1,1,0,0,1,0,1,1,1,0,0,1,1,1,0 \},$$ $$S_{A2}=\{1,1,0,0,0,1,0,0,1,0,1,0,0,0,1,1 \}.$$

    Finally, for $S_{B2}=S_{B1} \oplus S_{A1}$ Bob gets the following sequence:

    $$S_{B2}= \{1,1,0,0,1,1,0,1,0,1,0,0,1,1,0,1 \}.$$

    Note that the sets were enlarge in the second iteration.

  \item At this time, Bob sends again to Alice, through a classical channel, the minimum number between  ``ones'' and ``zeros'',  $n=min(\#0,\#1)$. In this case, $n$ is equal to $7$ which is the number of zeros.

  \item Alice checks if $n>s$ and acknowledge to Bob that she already knows that $n>s$. In this case, $n=7$ and $s=4$ being $n>s$ a valid condition.

  \item Next, Bob defines two new sub-sets, $I_{0}$ and $I_{1}$. $I_{0}$ is a set of values with photons array positions which Bob just guessed the measurement since he did not measure them with the same basis as Alice, $I_{1}$ is a set of values with photons array positions which Bob measured correctly since he used the same basis as Alice used to encoded them.

  In this example, Bob defines two sub-sets with size $s=4$:
  $$I_{0}=\{3,4,7,11 \},$$
  and $$I_{1}= \{2,5,6,13 \},$$ where $I_{0}$ is the sequence of positions in which Bob was wrong about basis measurement and $I_{1}$ is the sequence of positions in which Bob was right about basis measurement. Bob sends to Alice the set $S_{b}$

  Thus, if Bob wants to know $m_{0}$ he must send to Alice throughout a classical channel the set $S_{0}=\{I_{1},I_{0} \}$, otherwise if he wants to know $m_{1}$ he must send to Alice throughout a classical channel the set $S_{1}=\{I_{0},I_{1} \}$.


  \item With both the received set $S_{b}$ and the hash function value HASH1, Alice must be able to prove that Bob has being honest.

  \item Lets assume Bob sent $S_{0}=\{I_{1},I_{0} \}$.
   Alice defines two encryption keys $K_{0}$ and $K_{1}$ using the values in positions defined by Bob in the set sent by him. In this example, lets assume: $$K_{0}=\{1,0,1,0\}$$ $$K_{1}=\{0,0,0,1\}.$$

   Alice does the following operations:
   $$m = \{m_{0}\oplus K_{0}, m_{1} \oplus K_{1} \}.$$

   \begin{table}[H]
    \centering
    \begin{tabular}{c|c c c c c c c c}
     $m_{0}$ & 0 & 0 & 1 & 1 \\
     $K_{0}$ & 1 & 0 & 1 & 0 \\ \hline
     $\oplus$ & 1 & 0 & 0 & 1
    \end{tabular}
    \end{table}

   \begin{table}[H]
    \centering
    \begin{tabular}{c|c c c c c c c c}
     $m_{1}$ & 0 & 0 & 0 & 1 \\
     $K_{1}$ & 0 & 0 & 0 & 1 \\ \hline
     $\oplus$ & 0 & 0 & 0 & 0
    \end{tabular}
    \end{table}

    Adding the two results, $m$ will be: $$m=\{1,0,0,1,0,0,0,0\}.$$

   After that, Alice sends to Bob the encrypted message $m$ through a classical channel.

  \item When Bob receives the message $m$, in the same way as Alice, Bob uses $S_{B1\prime}$ values of positions given by $I_{1}$ and $I_{0}$ and does the decrypted operation. In this case, he does following operation:

      \begin{table}[H]
        \centering
        \begin{tabular}{c|c c c c c c c c}
         $m$ & 1 & 0 & 0 & 1 & 0 & 0 & 0 & 0 \\
             & 1 & 0 & 1 & 0 & 0 & 1 & 1 & 0 \\ \hline
         $\oplus$ & 0 & 0 & 1 & 1 & 0 & 1 & 1 & 0 \\
        \end{tabular}
        \end{table}

      The first four bits corresponds to message 1 and he received $\{0,0,1,1\}$, which is the right message $m_{0}$ and $\{0,1,1,0\}$ which is a wrong message for $m_{1}$.


\end{enumerate}

\subsection{Simulation Setup}

First of all, the protocol will be simulated and then a experimental setup will be built in the laboratory.

The main goal of this simulation is to demonstrate that Bob was able to learn correctly message $m_{b}$ and he does not know the message $m_{\overline{b}}$.

\begin{figure}[H]
	\centering
	\includegraphics[width=1.0\textwidth, height=9cm]{./sdf/bb84_with_discrete_variables/figures/Simulation_BB84.png}
	\caption{Simulation diagram at a top level}\label{toplevelsimulation}
\end{figure}
