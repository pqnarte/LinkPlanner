\clearpage
\section{BB84 with Discrete Variables}

\begin{tcolorbox}	
\begin{tabular}{p{2.75cm} p{0.2cm} p{10.5cm}} 	
\textbf{Students Name}  &:& Mariana Ramos and Kevin Filipe\\
\textbf{Starting Date} &:& November 7, 2017\\
\textbf{Goal}          &:& BB84 implementation with discrete variables.
\end{tabular}
\end{tcolorbox}

BB84 is a key distribution protocol which involves three parties, Alice, Bob and Eve. Alice and Bob exchange information between each other by using a quantum channel and a classical channel. The main goal is continuously build keys only known by Alice and Bob, and guarantee that eavesdropper, Eve, does not gain any information about the keys.


\subsection{Theoretical Description}

BB84 protocol was created by Charles Bennett and Gilles Brassard in 1984 \cite{BB84}. This was the first created Quantum Key Distribution (QKD) protocol. A basic model is depicted in figure \ref{fig:qkd model}. It involves two parties sharing keys through a quantum channel to decipher the classical channel data. It is assumed that there is a eavesdropper, Eve to intercept the data.

\begin{figure}[H]
	\centering
	\includegraphics[width=1.0\textwidth,height=7cm]{./sdf/bb84_with_discrete_variables/figures/QKD_Model.png}
	\caption{Basic QKD Model. Alice and Bob are connected by 2 communication channels, quantum and classical, with an eavesdropper, Eve, in the quantum communication channel. \cite{iqo}}\label{fig:qkd model}
\end{figure}


BB84 protocol uses bit encoding into photon state polarization. Two non-orthogonal basis are used to encoded the information, the rectilinear and diagonal basis. The following table shows this bit encoding and figure \ref{fig:basis} shows a simple representation of it.

\begin{table}[H]
	\centering
	\begin{tabular}{c|c|c}
		 &  \textbf{\textit{Rectilinear Basis}} & \textbf{\textit{Diagonal Basis}}\\ \hline
		0 &  0$º$ & -45$º$ \\
		1 & 90$º$ & 45$º$\\
	\end{tabular}
\end{table}

\begin{figure}[H]
	\centering
	\includegraphics[width=0.4\textwidth,height=4cm]{./sdf/bb84_with_discrete_variables/figures/basis.png}
	\caption{Simple representation of the bit encoding using the corresponding bases.\cite{SURV}}\label{fig:basis}
\end{figure}


The protocol is implemented with the following steps:

\begin{enumerate}
	\item Alice generates two random bit strings. The random string ,$R_{A1}$, is the data to be encoded while $R_{A2} $is a string which 0 and 1 corresponds to rectilinear and diagonal basis $B_{A}$, respectively.
	
	$$ R_{A1} = \{0,1,1,0,1,0,0,1,1,0,1,1,1,0,0,1,0,0,0,1\}$$
	$$ R_{A2} = \{0,0,1,0,1,1,1,0,1,1,1,0,1,0,0,0,1,0,1,0\}$$
	$$ B_{A} = \{+,+,\times,+,\times, \times, \times, +,\times, \times, \times,+,\times,+,+,+,\times,+,\times,+\}$$
	
	\item Alice transmits a train of photons,$S_{AB}$, in which, each photon represents a bit of the random string, Ra1, with the correspondent polarization regarding to the bit value.
	
	$$S_{AB} = \{\to, \uparrow, \searrow, \to, \searrow, \nearrow, \nearrow, \uparrow, \searrow, \nearrow, \searrow, \uparrow, \searrow, \to, \to, \uparrow, \nearrow, \to, \nearrow, \uparrow\}.$$
	
	\item Bob generates a random string, $R_{B}$, such as Alice, to receive the photon trains with the correspondent basis.
	
	$$ R_{B} = \{0,1,1,1,0,1,0,0,1,1,0,0,1,1,0,0,1,1,0,0\}$$
	$$B_{B} = \{ +,\times,\times,\times,+,\times,+,+,\times,\times,+,+,\times,\times,+,+,\times,\times,+,+\}.$$
	
	\item If Bob chooses a matched basis compared to the one encoded in the photon, then he can correctly deduce the right bit, otherwise deduced bit is randomly read. It is considered that the channel contains attenuation, this will not click the polarization detector
	
	$$D_{B} = \{0,1,1,0,-,0,1,1,-,0,0,1,-,1,0,1,0,-,1,1\}$$
	
\end{enumerate}

Since less than half of the data is deduced, a second phase is needed so that Alice can announce to Bob which deduced bits are correct This second phase, uses the classical communication channel:
\begin{enumerate}
	
	\item Bob sends Alice about the no clicks, in which 1 represents a no click.
	
	$$NC_{BA} = \{ 0,0,0,0,1,0,0,0,1,0,0,0,1,0,0,0,0,1,0,0\}.$$
	
	\item Bob notifies Alice about what random basis he used to deduce each bit, $R_{B}$, and Alice also sends to Bob the basis $R_{A1}$, she used.
	
	\item Both performs an XOR to deduce the correct sequence.
	
	\begin{table}[H]
		\centering
		\begin{tabular}{c|c c c c c c c c c c c c c c c c c c c c}
			$R_{A2}$ & 0 & 0 & 1 & 0 & 1 & 1 & 1 & 0 & 1 & 1 & 1 & 0 & 1 & 0 & 0 & 0 & 1 & 0 & 1 & 0 \\
			$R_{B}$  & 0 & 1 & 1 & 1 & 0 & 1 & 0 & 0 & 1 & 1 & 0 & 0 & 1 & 1 & 0 & 0 & 1 & 1 & 0 & 0 \\ \hline
			$\oplus$ & 0 & 1 & 0 & 1 & 1 & 0 & 1 & 0 & 0 & 0 & 1 & 0 & 0 & 1 & 0 & 0 & 0 & 1 & 1 & 0 \\
		\end{tabular}
	\end{table}
	
	\item By using the deducid bits and removing the no clicks, the final obtained sequence is $K_{AB}$
	
  	$$ K_{AB} = \{0,1,0,1,0,1,0,1,0,1\}.$$
		
\end{enumerate}

	The Quantum Bit Error Rate (QBER) is calculated by knowing the amount of no clicks and correct deduced bits by Bob.
	
	$$ QBER = no\_clicks\_qty / correct\_deduced\_bits $$
	
	In the previous example, the QBER is 33$\%$, since there are 4 no clicks.	

The presence of a eavesdropper will carry the risk of changing the bits. This will produce disagreement between Bob and Alice in the bits they should agree. When Eve measures and retransmits a photon she can deduce correctly with a probability of 50\%. So by learning the correct polarization of half of the photons, the induced error is 25\%. Alice and Bob can detect Eve presence by sacrifice the secrecy of some bits in order to test.

\begin{thebibliography}{2}
\bibitem{BB84}
Bennett, C. H. and Brassard,
G. Quantum Cryptography: Public key distribution and coin tossing.
International Conference on Computers, Systems and Signal Processing, Bangalore, India, 10-12 December 1984, pp. 175-179.

\bibitem{SURV}
Mart Haitjema, A Survey of the Prominent Quantum Key Distribution Protocols

\bibitem{iqo}
Christopher Gerry, Peter Knight, "Introductory Quantum Optics" Cambridge University Press, 2005

\end{thebibliography}
\cleardoublepage

\begin{table}[H]
\centering
\begin{tabular}{c|c}
\textbf{\textit{Basis}}         &  \\ \hline
 0 & $+$ \\
 1 & $\times$ \\
\end{tabular}
\end{table}


\begin{table}[H]
\centering
\begin{tabular}{c|c}
            & Basis "+" \\ \hline
 0 & $\to (0^{\circ})$ \\
 1 & $\uparrow (90^{\circ})$ \\
\end{tabular}
\end{table}


\begin{table}[H]
\centering
\begin{tabular}{c|c}
      & Basis "$\times$" \\ \hline
 0 & $\searrow (-45^{\circ})$ \\
 1 & $\nearrow (45^{\circ})$ \\
\end{tabular}
\end{table}


\begin{enumerate}
  \item Alice randomly generate a bit sequence with length $ks$ being, in this case, $k=2$ and $s=4$ as it was defined at the beginning.
      Therefore, she must define two sets randomly: $S_{A1}$ which contains the basis values; and $S_{A2}$, which contains the key values.

      In that case, lets assume she gets the following sets $S_{A1}$ and $S_{A2}$:
      $$S_{A1} = \{0,1,1,0,0,1,0,1 \},$$
      $$S_{A2} = \{1,1,0,0,0,1,0,0 \}.$$

  \item Next, Alice sends to Bob throughout a quantum channel $ks$ photons encrypted using the basis defined in $S_{A1}$ and according to the keys defined in $S_{A2}$.

      In the current example, Alice sends the photons, throughout a quantum channel, according to the following,

      $$S_{AB} = \{\uparrow, \nearrow, \searrow, \to, \to, \nearrow, \to, \searrow \}.$$
       $$S_{AB} = \{90^{\circ}, 45^{\circ}, -45^{\circ}, 0^{\circ}, 0^{\circ}, 45^{\circ}, 0^{\circ}, -45^{\circ} \}.$$

  \item Bob also randomly generates $ks$ bits, which are going to define his measurement basis, $S_{B1}$. He will measure the photons sent by Alice. Lets assume:

  $$S_{B1} = \{0,1,0,1,0,1,1,1 \}.$$

    When Bob receives photons from Alice, he measures them using the basis defined in $S_{B1}$.
  In the current example, $S_{B1}$ corresponds to the following set:
  $$\{ +,\times,+,\times,+,\times, \times, \times \}.$$
  Bob will get $ks$ results:
  $$S_{B1'} = \{1,1,0,1,0,1,1,0 \}.$$

  \item Bob will send a \textit{Hash Function} result HASH1 to Alice. This value will do Bob's commitment with the measurements done. In this case, this \textit{Hash Function} is calculated from \textit{SHA-256} algorithm for each pair (Basis from $S_{B1}$ and measured value from $S_{B1'}$), i.e Bob sends to Alice $sk$ pairs as his commitment. In this case, Bob sends eight pairs encoded using a \textit{Hash Function} which is also send to Alice. From that moment on Bob cannot change his commitment neither the basis which he uses to measure the photons sent by Alice.

  \item Once Alice has received the confirmation of measurement from Bob, she sends throughout a classical channel the basis which she has used to codify the photons, which in this case we assumed $S_{A1} = \{0,1,1,0,0,1,0,1\}$.

  \item In order to know which photons were measured correctly, Bob does the operation $S_{B2}=S_{B1} \oplus S_{A1}$.
      In the current example the operation will be:

  \begin{table}[H]
    \centering
    \begin{tabular}{c|c c c c c c c c}
     $S_{B1}$ & 0 & 1 & 0 & 1 & 0 & 1 & 1 & 1 \\
     $S_{A1}$ & 0 & 1 & 1 & 0 & 0 & 1 & 0 & 1 \\ \hline
     $\oplus$ & 1 & 1 & 0 & 0 & 1 & 1 & 0 & 1
    \end{tabular}
    \end{table}

      In this way, Bob gets $$S_{B2} = \{1,1,0,0,1,1,0,1 \}.$$ When Bob uses the right basis he gets the values correctly, when he uses the wrong basis he just guess the value. The values ``$1$'' correspond to the values he measured correctly and ``$0$'' to the values he just guessed.

       Next, Bob sends to Alice, through a classical channel, information about the minimum number between ``ones'' and ``zeros'', i.e $$n=min(\#0,\#1)=3,$$ where $\#0$ represents the number of zeros in $S_{B2}$ and $\#1$ the number of ones in $S_{B2}$. At this time, Alice must be able to know if Bob is being honest or not. Therefore, she will open Bob's commitment from \textit{step 4} and she verify if the number n sent by Bob is according with the commitment values sent by him. In other words, she opens a number of pairs committed by Bob which is known from the beginning.

  \item If $n<s$, being \textit{s} the message's size, Alice and Bob will repeat the steps from $1$ to $7$. In this case, $n=3$ which is smaller than $s=4$. Therefore, Alice and Bob repeat the steps from 1 to 7 in order to enlarge Bob's sets $S_{B1}$ and $S_{B2}$ as well as Alice's sets $S_{A1}$ and $S_{A2}$.

  \item Lets assume :

   $$S_{B1}= \{1,1,0,0,0,1,0,0,1,0,0,0,0,0,1,1 \}.$$

    At Alice's side the new sets $S_{A1}$, which contains the basis values, and $S_{A2}$, which contains the key values, will be the following:

    $$S_{A1}=\{0,1,1,0,0,1,0,1,1,1,0,0,1,1,1,0 \},$$ $$S_{A2}=\{1,1,0,0,0,1,0,0,1,0,1,0,0,0,1,1 \}.$$

    Finally, for $S_{B2}=S_{B1} \oplus S_{A1}$ Bob gets the following sequence:

    $$S_{B2}= \{1,1,0,0,1,1,0,1,0,1,0,0,1,1,0,1 \}.$$

    Note that the sets were enlarge in the second iteration.

  \item At this time, Bob sends again to Alice, through a classical channel, the minimum number between  ``ones'' and ``zeros'',  $n=min(\#0,\#1)$. In this case, $n$ is equal to $7$ which is the number of zeros.

  \item Alice checks if $n>s$ and acknowledge to Bob that she already knows that $n>s$. In this case, $n=7$ and $s=4$ being $n>s$ a valid condition.

  \item Next, Bob defines two new sub-sets, $I_{0}$ and $I_{1}$. $I_{0}$ is a set of values with photons array positions which Bob just guessed the measurement since he did not measure them with the same basis as Alice, $I_{1}$ is a set of values with photons array positions which Bob measured correctly since he used the same basis as Alice used to encoded them.

  In this example, Bob defines two sub-sets with size $s=4$:
  $$I_{0}=\{3,4,7,11 \},$$
  and $$I_{1}= \{2,5,6,13 \},$$ where $I_{0}$ is the sequence of positions in which Bob was wrong about basis measurement and $I_{1}$ is the sequence of positions in which Bob was right about basis measurement. Bob sends to Alice the set $S_{b}$

  Thus, if Bob wants to know $m_{0}$ he must send to Alice throughout a classical channel the set $S_{0}=\{I_{1},I_{0} \}$, otherwise if he wants to know $m_{1}$ he must send to Alice throughout a classical channel the set $S_{1}=\{I_{0},I_{1} \}$.


  \item With both the received set $S_{b}$ and the hash function value HASH1, Alice must be able to prove that Bob has being honest.

  \item Lets assume Bob sent $S_{0}=\{I_{1},I_{0} \}$.
   Alice defines two encryption keys $K_{0}$ and $K_{1}$ using the values in positions defined by Bob in the set sent by him. In this example, lets assume: $$K_{0}=\{1,0,1,0\}$$ $$K_{1}=\{0,0,0,1\}.$$

   Alice does the following operations:
   $$m = \{m_{0}\oplus K_{0}, m_{1} \oplus K_{1} \}.$$

   \begin{table}[H]
    \centering
    \begin{tabular}{c|c c c c c c c c}
     $m_{0}$ & 0 & 0 & 1 & 1 \\
     $K_{0}$ & 1 & 0 & 1 & 0 \\ \hline
     $\oplus$ & 1 & 0 & 0 & 1
    \end{tabular}
    \end{table}

   \begin{table}[H]
    \centering
    \begin{tabular}{c|c c c c c c c c}
     $m_{1}$ & 0 & 0 & 0 & 1 \\
     $K_{1}$ & 0 & 0 & 0 & 1 \\ \hline
     $\oplus$ & 0 & 0 & 0 & 0
    \end{tabular}
    \end{table}

    Adding the two results, $m$ will be: $$m=\{1,0,0,1,0,0,0,0\}.$$

   After that, Alice sends to Bob the encrypted message $m$ through a classical channel.

  \item When Bob receives the message $m$, in the same way as Alice, Bob uses $S_{B1\prime}$ values of positions given by $I_{1}$ and $I_{0}$ and does the decrypted operation. In this case, he does following operation:

      \begin{table}[H]
        \centering
        \begin{tabular}{c|c c c c c c c c}
         $m$ & 1 & 0 & 0 & 1 & 0 & 0 & 0 & 0 \\
             & 1 & 0 & 1 & 0 & 0 & 1 & 1 & 0 \\ \hline
         $\oplus$ & 0 & 0 & 1 & 1 & 0 & 1 & 1 & 0 \\
        \end{tabular}
        \end{table}

      The first four bits corresponds to message 1 and he received $\{0,0,1,1\}$, which is the right message $m_{0}$ and $\{0,1,1,0\}$ which is a wrong message for $m_{1}$.


\end{enumerate}

\newpage

\subsection{Simulation Analysis}

\begin{tcolorbox}	
\begin{tabular}{p{2.75cm} p{0.2cm} p{10.5cm}} 	
\textbf{Students Name}  &:& Mariana Ramos \\
\textbf{Starting Date} &:& November 7, 2017\\
\textbf{Goal}          &:& Perform a simulation of the setup presented bellow in order to implement BB84 communication protocol.
\end{tabular}
\end{tcolorbox}

In this sub section the simulation setup implementation will be described in order to implement the BB84 protocol. In figure \ref{toplevelsimulation} a top level diagram is presented. Then it will be presented the block diagram of the transmitter block (Alice) in figure \ref{alicesimulation}, the receiver block (Bob) in figure \ref{bobsimulation} and finally the eavesdropper block (Eve) in figure \ref{evesimulation}.

\begin{figure}[H]
	\centering
	\includegraphics[width=1.0\textwidth, height=9cm]{./sdf/bb84_with_discrete_variables/figures/toplevel_simulation.png}
	\caption{Simulation diagram at a top level}\label{toplevelsimulation}
\end{figure}

Figure \ref{toplevelsimulation} presents the top level diagram of our simulation. The setup contains three parties, Alice, Eve and Bob where the communication between them is done throughout two classical and one quantum channel. In the middle of the classical channel there is a Fork's diagram which has one input and two outputs. In the case of the classical channel \textbf{C\_C\_4} which has the information sent by Bob, the fork's block enables Alice and Eve have access to it. In the quantum communication, the information sent by Alice can be intercepted by Eve and changed by her, or can follow directly to Bob as we can see later in figure \ref{evesimulation}. Furthermore, for mutual information calculation there must be two blocks \textbf{MI}, one to calculate the mutual information between Alice and Eve, and other to calculate the mutual information between Alice and Bob.

\begin{figure}[h]
	\centering
	\includegraphics[width=1.0\textwidth, height=9cm]{./sdf/bb84_with_discrete_variables/figures/alice_simulation.png}
	\caption{Simulation diagram at Alice's side}\label{alicesimulation}
\end{figure}

\begin{figure}[h]
	\centering
	\includegraphics[width=1.0\textwidth, height=9cm]{./sdf/bb84_with_discrete_variables/figures/bob_simulation.png}
	\caption{Simulation diagram at Bob's side}\label{bobsimulation}
\end{figure}

    In figure \ref{alicesimulation} one can observe a block diagram of the simulation at Alice's side. As it is shown in the figure, Alice must have one block for random number generation which is responsible for basis generation to polarize the photons, and for key random generation in order to have a random state to encode each photon. Furthermore, she has a Processor block for all logical operations: array analysis, random number generation requests, and others. This block also receives the information from Bob after it has passed through a fork's block. In addition, it is responsible for set the initial length $l$ of the first array of photons which will send to Bob. This block also must be responsible for send classical information to Bob. Finally, Processor block will also send a real continuous time signal to single photon generator, in order to generate photons according to this signal, and finally this block also sends to the polarizer a real discrete signal in order to inform the polarizer which basis it should use. Therefore, she has two more blocks for quantum tasks: the single photon generator and the polarizer block which is responsible to encode the photons generated from the previous block and send them throughout a quantum channel from Alice to Bob.

    Finally, Alice's processor has an output to Mutual Information top level block, $Ms_{A}$.

     In figure \ref{bobsimulation} one can observe a block diagram of the simulation at Bob's side. From this side, Bob has one block for Random Number Generation which is responsible for randomly generate basis values which Bob will use to measure the photons sent by Alice throughout the quantum channel. Like Alice, Bob has a Processor block responsible for all logical tasks, analysing functions, requests for random number generator block, etc. Additionally, it receives information from Alice throughout a classical channel after passed through a fork's block and a quantum channel. However, Bob only sends information to Alice throughout a classical channel. Furthermore, Bob has one more block for single photon detection which receives from processor block a real discrete time signal, in order to obtain the basis it should use to measure the photons.

    Finally, Bob's processor has an output to Mutual Information top level block, $Ms_{B}$.



\begin{figure}[h]
	\centering
	\includegraphics[width=1.1\textwidth, height=14cm]{./sdf/bb84_with_discrete_variables/figures/eve_simulation.png}
	\caption{Simulation diagram at Eve's side}\label{evesimulation}
\end{figure}

Figure \ref{evesimulation} presents the Eve's side diagram. Eve's processor has two receiver classical signals, one from Alice (\textbf{C\_C\_2}) and other from Bob (\textbf{C\_C\_5}). About quantum channel, Eve received a quantum message from Alice through the channel \textbf{Q\_C\_1} and depends on her decision the photon can follows directly to Bob or the photon's state can be changed by her. In this case, the photon is received by a block similar to Bob's diagram \ref{bobsimulation} and this block sends a message to Eve's processor in order to reveal the measurement result. After that, Eve's processor sends a message to Alice's diagram similar to figure \ref{alicesimulation} and this block is responsible for encode the photon in a new state. Now, the changed photon is sent to Bob.

In addition, Eve's diagram has one more output $Ms_{E}$ which is a message sent to the mutual information block as an input parameter.

\begin{table}[H]
\centering
\caption{System Signals}
\label{tb:signals}
\begin{tabular}{|c|c|c|}
\hline
\textbf{Signal name}            & \textbf{Signal type}                      & \textbf{Status} \\ \hline
NUM\_A                          &  Binary                                   &                 \\ \hline
MI\_A                           &  Binary                                   &                 \\ \hline
CLK\_A                          &  TimeContinuousAmplitudeContinuous        &                 \\ \hline
CLK\_A\_1                       &  TimeContinuousAmplitudeContinuous        &                 \\ \hline
S2                              &  PhotonStreamXY                           &                 \\ \hline
S3                              &  TimeContinuousAmplitudeDiscreteReal      &                 \\ \hline
C\_C\_1                         &  Messages                                 &                 \\ \hline
C\_C\_6                         &  Messages                                 &                 \\ \hline
Q\_C\_1                         &  PhotonStreamXY                           &                 \\ \hline

\end{tabular}
\end{table}

Table \ref{tb:signals} presents the system signals as well as them type.

\begin{table}[H]
\centering
\caption{System Input Parameters}
\label{tb:inputparameters}
\begin{tabular}{|c|c|c|}
\hline
\textbf{Parameter}                      & \textbf{Default Value}                                & \textbf{Description} \\ \hline
RateOfPhotons                           & 1K                                                    &                 \\ \hline
vector<t\_iqValues> iqAmplitudeValues   & \{-45.0,0.0\},\{0.0,0.0\},\{45.0,0.0\},\{90.0,0.0\}   &                 \\ \hline

\end{tabular}
\end{table}

\begin{table}[H]
\centering
\caption{Header Files}
\label{tb:signals}
\begin{tabular}{|c|c|c|}
\hline
\textbf{File name}              & \textbf{Description} & \textbf{Status} \\ \hline
netxpto.h                             &                      &    \checkmark   \\ \hline
alice\_qkd.h                          &                      &  Working on     \\ \hline
binary\_source.h                      &                      &    \checkmark   \\ \hline
bob\_qkd.h                            &                      &   Missing       \\ \hline
clock\_20171219.h                     &                      &    \checkmark   \\ \hline
discrete\_to\_continuous\_time.h      &                      &    \checkmark   \\ \hline
eve\_qkd.h                            &                      &   Missing       \\ \hline
m\_qam\_mapper.h                      &                      &    \checkmark   \\ \hline
optical\_switch.h                     &                      &   Missing       \\ \hline
polarization\_beam\_splitter.h        &                      &  Working on     \\ \hline
polarizer.h                           &                      &  Working on     \\ \hline
pulse\_shaper.h                       &                      &     \checkmark  \\ \hline
rotator\_linear\_polarizer.h          &                      &  Working on     \\ \hline
single\_photon\_detector.h            &                      &   Missing       \\ \hline
single\_photon\_source\_20171218.h    &                      &    \checkmark   \\ \hline
sink.h                                &                      &    \checkmark   \\ \hline
super\_block\_interface.h             &                      &    \checkmark   \\ \hline
\end{tabular}
\end{table}

\begin{table}[H]
\centering
\caption{Source Files}
\label{tb:signals}
\begin{tabular}{|c|c|c|}
\hline
\textbf{File name}                      & \textbf{Description} & \textbf{Status} \\ \hline
netxpto.cpp                             &                      &    \checkmark   \\ \hline
bb84\_with\_discrete\_variables.cpp     &                      &  Working on     \\ \hline
alice\_qkd.cpp                          &                      &  Working on     \\ \hline
binary\_source.cpp                      &                      &    \checkmark   \\ \hline
bob\_qkd.cpp                            &                      &   Missing       \\ \hline
clock\_20171219.cpp                     &                      &    \checkmark   \\ \hline
discrete\_to\_continuous\_time.cpp      &                      &    \checkmark   \\ \hline
eve\_qkd.cpp                            &                      &   Missing       \\ \hline
m\_qam\_mapper.cpp                      &                      &    \checkmark   \\ \hline
optical\_switch.cpp                     &                      &   Missing       \\ \hline
polarization\_beam\_splitter.cpp        &                      &  Working on     \\ \hline
polarizer.cpp                           &                      &  Working on     \\ \hline
pulse\_shaper.cpp                       &                      &     \checkmark  \\ \hline
rotator\_linear\_polarizer.cpp          &                      &  Working on     \\ \hline
single\_photon\_detector.cpp            &                      &   Missing       \\ \hline
single\_photon\_source\_20171218.cpp    &                      &    \checkmark   \\ \hline
sink.cpp                                &                      &    \checkmark   \\ \hline
super\_block\_interface.cpp             &                      &    \checkmark   \\ \hline
\end{tabular}
\end{table} 

\subsection{Open Issues}

There still are some open issues in simulation code.

One of them was detected in block \textbf{single\_photon\_source\_20171218.cpp}. This block should assume each sample with 4 real values, since it writes two complex values each time the block runs, i.e each \textit{bufferput()} should write an array of two complex values in outputSignal, outputSignals[0]->bufferPut(valueXY), where \textit{t\_complex\_xy valueXY = \{valueX, valueY\}} and \textit{t\_complex valueX = (realValue\_1,realValue\_2)}. This way, independently of the number of samples these four values should always be written. However, if we chose a number of samples which is not divisible by 4, the four numbers are not written in the last "sample" and the array data for X's values and Y's values have different sizes which is wrong. For example, if we chose 10 samples to acquire, the last values correspond to X's values instead of Y's values and the first array data is longer than the other. 