\clearpage

\section{Single Photon Receiver}

\begin{tcolorbox}	
\begin{tabular}{p{2.75cm} p{0.2cm} p{10.5cm}} 	
\textbf{Header File}   &:& single\_photon\_receiver\_20180303.h \\
\textbf{Source File}   &:& single\_photon\_receiver\_20180303.cpp \\
\textbf{Version}       &:& 20180303 (\textbf{Student Name}: Mariana Ramos)
\end{tabular}
\end{tcolorbox}

This block of code simulates the reception of two time continuous signals which are the outputs of single photon detectors and decode them in measurements results. A simplified schematic representation of this block is shown in figure \ref{SPR_receiver_block_diagram_simple}.

\begin{figure}[h]
	\centering
	\includegraphics[clip, trim=8cm 4cm 6cm 5cm, width=1.00\textwidth]{../lib/single_photon_receiver/figures/single_photon_receiver.pdf}
	\caption{Basic configuration of the SPR receiver}\label{SPR_receiver_block_diagram_simple}
\end{figure}

\subsection*{Functional description}

This block accepts two time continuous input signals and outputs one time discrete signal that corresponds to the single photon detection measurements demodulation of the input signal. It is a complex block (as it can be seen from figure \ref{SPR_receiver_block_diagram_simple} of code made up of several simpler blocks whose description can be found in the \textit{lib} repository.

In can also be seen from figure \ref{SPR_receiver_block_diagram_simple} that there are two extra internal input signals generated by the \textit{Clock} in order to keep all blocks synchronized. This block is used to provide the sampling frequency to the \textit{Sampler} blocks.


\subsection*{Input parameters}

This block has some input parameters that can be manipulated by the user in order to change the basic configuration of the receiver. Each parameter has associated a function that allows for its change. In the following table (table~\ref{table:spr_in_par}) the input parameters and corresponding functions are summarized.

\begin{table}[h]
	\centering
	\begin{tabular}{|c|c|c|c|cccc}
		\cline{1-4}
		\textbf{Parameter} & \textbf{Type} & \textbf{Values} &   \textbf{Default}& \\ \cline{1-4}
		samplesToSkip & int & any & 0 \\ \cline{1-4}
		filterType  & PulseShaperFilter & RaisedCosine, Gaussian,  & Square \\ 
                    &                   & Square, RootRaisedCosine &        \\ \cline{1-4}
	
	\end{tabular}
	\caption{List of input parameters of the block SOP modulator}
	\label{table:spr_in_par}
\end{table}


\subsection*{Methods}

SinglePhotonReceiver(vector <Signal*> \&inputSignals, vector <Signal*> \&outputSignals)(\textbf\{constructor\})
\bigbreak
void setPulseShaperFilter(PulseShaperFilter fType)
\bigbreak
void setPulseShaperWidth(double pulseW)
\bigbreak
void setClockBitPeriod(double period)
\bigbreak
void setClockPhase(double phase)
\bigbreak
void setClockSamplingPeriod(double sPeriod)
\bigbreak
void setThreshold(double threshold)

\subsection*{Input Signals}

\subparagraph*{Number:} 2

\subparagraph*{Type:} Time Continuous Amplitude Continuous Real

\subsection*{Output Signals}

\subparagraph*{Number:} 1

\subparagraph*{Type:} Time Discrete Amplitude Discrete Real

\subsection*{Example}

\subsection*{Sugestions for future improvement}
